\documentclass{article}
\usepackage[utf8]{inputenc}
\usepackage{amsmath, amsfonts, amssymb,amsthm, mathtools,xcolor, algorithm, algorithmicx, dsfont, cite}
\usepackage[noend]{algpseudocode}

\newcommand{\note}[1]{\textcolor{red}{#1}}
\newcommand{\opt}{\mathrm{OPT}}
\newcommand{\In}{\mathrm{In}}
\newcommand{\triv}{\mathrm{Triv}}
\newcommand{\ntriv}{\mathrm{NonTriv}}
\DeclareMathOperator*{\argmin}{argmin}
\DeclareMathOperator*{\argmax}{argmax}
\DeclareMathOperator*{\RF}{RF}
\newcommand{\bs}{bisup}
\DeclareMathOperator*{\extra}{Extra}
\setlength\parindent{0pt}
\newtheorem{theorem}{Theorem}
\newtheorem{lemma}{Lemma}
\newtheorem{claim}{Claim}
\newtheorem{corollary}{Corollary}
\newtheorem{definition}{Definition}

\title{Minimum Robinson-Foulds Distance Supertree}
\author{Xilin Yu, Thien Le, Sarah Christensen,  Erin Molloy, Tandy Warnow}
\date{\today}

\begin{document}

\maketitle

Throughout the paper, we consider only unrooted trees. For any tree $T$, let $V(T)$, $E(T)$, and $L(T)$ denote the vertex set, the edge set, and the leaf set of $T$, respectively. For any $v\in V(T)$, let $N_T(v)$  A tree is \textit{fully resolved} if every non-leaf node has degree $3$. Let $\mathcal{T}_S$ denote the set of all fully resolved trees on leaf set $S$. In any tree $T$, each edge $e$ induces a bipartition $\pi_e := A|B$ of the leaf set, where $A$ and $B$ are the leaves in the two components of $T-e$, respectively. A bipartition $A|B$ is non-trivial if both sides have size at least $2$. For a tree $T$, $C(T) := \{\pi_e \mid e\in E(T)\}$ denotes the set of all bipartitions of $T$. For a fully resolved tree with $n$ leaves, $C(T)$ contains $2n-3$ bipartitions, exactly $n-3$ of which are non-trivial. A tree $T'$ is a \textit{refinement} of $T$ if $T$ can be obtained from $T'$ by contracting a set of edges. Equivalently, $T'$ is a refinement of $T$ if and only if $C(T) \subseteq C(T')$.\\

Two bipartitions $\pi_1$ and $\pi_2$ of the same leaf set are \textit{compatible} if and only if there exists a tree $T$ such that $\pi_1, \pi_2 \in C(T)$. The following theorem and corollary give other categorizations of compatibility.

\begin{theorem}[Theorem 2.20 of \cite{warnow2017computational}]\label{thm:compatibility}
    A pair of bipartitions $A|B$ and $A'|B'$ of the same set is compatible if and only if at least one of the four pairwise intersections $A \cap A'$, $A\cap B'$, $B\cap A'$, $B \cap B'$ is empty. 
\end{theorem}

\begin{corollary}\label{cor:compatibility}
     A pair of bipartitions $A|B$ and $A'|B'$ of the same set is compatible if and only if one side of $A|B$ is a subset of one side of $A'|B'$.
\end{corollary}

A tree $T$ restricted to a subset $R$ of its leaf set, denoted $T|_R$, is the minimal subtree of $T$ spanning $R$ with nodes of degree two suppressed. A bipartition $\pi = A|B$ restricted to a subset $R \subseteq A\cup B$ is $\pi|_R = A\cap R | B\cap R$. We have the following intuitive lemma with its proof in the appendix.

\begin{lemma} \label{lem:bipar_restrict_edge}
    Let $T$ be a tree with leaf set $S$ and let $\pi = A|B \in C(T)$ be a bipartition induced by $e \in E(T)$. Let $R \subseteq S$.
    \begin{enumerate}
        \item If $R \cap A = \emptyset$ or $R \cap B = \emptyset$, then $e \notin E(T|_R)$.
        \item If $R \cap A \neq \emptyset$ and $R \cap B \neq \emptyset$, then for any $\pi' \in C(T|_R)$ induced by $e' \in E(T|_R)$, $\pi|_R = \pi'$ if and only if $e \in P(e')$.
    \end{enumerate}
\end{lemma}

\begin{corollary} \label{cor:bipar_restrict}
    Let $T$ be a tree with leaf set $S$ and let $\pi = A|B \in C(T)$ be a bipartition induced by $e \in E(T)$. Let $R \subseteq S$ such that $R \cap A \neq \emptyset$ and $R \cap B \neq \emptyset$. Then $\pi|_R \in C(T|_R)$. 
\end{corollary}

\begin{definition}
For two trees $T$, $T'$ with the same leaf set, the \textit{bipartition support} of them is $\bs(T, T') := |C(T) \cap C(T')|$.
\end{definition}

Let $T_1$ and $T_2$ be two fully resolved trees on leaf sets $S_1$ and $S_2$, respectively, such that $X := S_1 \cap S_2 \neq \emptyset$. Let $S := S_1 \cup S_2$. The Maximum Bipartition Support Supertree problem, abbreviated \textsc{Max-Bisup-Supertree}, finds a fully resolved supertree $T^*$ on leaf set $S$ that maximizes the sum of the bipartition support of $T^*$ with respect to $T_1$ and $T_2$. That is, 
\begin{align*}
    T^* &= \argmax_{T \in \mathcal{T}_S} \bs(T|_{S_1}, T_1) + \bs(T|_{S_2}, T_2)\\ 
        &= \argmax_{T \in \mathcal{T}_S} |C(T|_{S_1})\cap C(T_1)| + |C(T|_{S_2}) \cap C(T_2)|.
\end{align*}

We call $\bs(T|_{S_1}, T_1) + \bs(T|_{S_2}, T_2)$ the support score of $T$ when $T_1$ and $T_2$ are clear from context.\\

We first set up the notations for the algorithm and the analysis. Let $T_1,T_2,S_1,S_2$, and $X$ be defined as from the problem statement. Let $T_1|_X$ and $T_2|_X$ be the backbone trees of $T_1$ and $T_2$, respectively. Let $\Pi$ be the set of bipartitions of $X$. Let $\triv$ and $\ntriv$ denotes the set of trivial and non-trivial bipartitions in $C(T_1|_X) \cup C(T_2|_X)$. For each $e \in E(T_i|_X)$, $i \in \{1,2\}$, let $P(e)$ denote the path in $T_i$ from which $e$ is obtained by suppressing all degree-two nodes. Let $w(e)$ be the number of edges on $P(e)$. \\

We define a weight function $w:\Pi \to \mathbb{N}_{\ge 0}$ such that for any bipartition $\pi$ of $X$, $w(\pi) = w(e_1) + w(e_2)$, where $e_i$ induces $\pi$ in $T_i|_X$ for $i \in \{1,2\}$. If for any $i \in \{1,2\}$, no $e_i$ exists that induces $\pi$ in $T_i|_X$, then we use $w(e_i) = 0$.\\

For each $i \in \{1,2\}$ and each $e \in E(T_i|_X)$, let $\In(e)$ be the set of internal nodes of $P(e)$. For each $v \in \In(e)$, let $L(v)$ be the set of leaves in $S_i \backslash X$ whose connecting path to the backbone tree $T_i|_X$ goes through $v$ and let $T(v)$ be the minimal subtree spanning $L(v)$ in $T_i$. We say $T(v)$ is an extra subtree attached to $v$. We let the node $u$ which is the neighbor of $v$ in $T(v)$ be the root of $T(v)$. Let $\mathcal{T}(e) := \{T(v) \mid v \in \In(e)\}$. Then $\mathcal{T}(e)$ is the set of extra subtrees attached to internal nodes of $P(e)$ in $T_i$. We note that $|\mathcal{T}(e)| = |\In(e)| = w(e)-1$. For any bipartition $\pi \in C(T_1|_X) \cup C(T_2|_X)$, we denote $\mathcal{T}(\pi) := \mathcal{T}(e_1) \cup \mathcal{T}(e_2)$, where $e_i$ is the edge that induces $\pi$ in $T_i|_X$ for $i \in \{1,2\}$ if $\pi \in C(T_i|_X)$. Let $\extra(T_i) := \bigcup_{e \in E(T_i|_X)} \mathcal{T}(e)$. Then $\extra := \extra(T_1) \cup \extra(T_2)$ denotes the set of all extra subtrees in $T_1$ and $T_2$.\\


\begin{algorithm}
    \caption{Max-BiSup Supertree}
    \label{alg:maxbisup}
    \textbf{Input}: two fully resolved trees $T_1$, $T_2$ with leaf sets $S_1$ and $S_2$ where $S_1 \cap S_2 = X \neq \emptyset$\\
    \textbf{Output}: a fully resolved supertree $T$ on leaf set $S = S_1 \cup S_2$ that maximizes the support score 
    \begin{algorithmic}[1]
        \State compute $C(T_1|_X)$ and $C(T_2|_X)$
        \For{each $\pi \in C(T_1|_X) \cup C(T_2|_X)$}
            \State compute $\mathcal{T}(\pi)$ and $w(\pi)$
        \EndFor
        \State construct $T$ by having a star of leaf set $X$ with center vertex $\hat{v}$ and connecting the root of each $t \in \extra$ to $\hat{v}$, let $\hat{T} = T$       
        \For{each $\pi \in \triv$}
            \State $T \gets $ Refine-Triv($T_1, T_2, T, \pi, \hat{v}, \mathcal{T}$)
        \EndFor
        \State construct the incompatibility graph $G = (V_1 \cup V_2, E)$, where $V_1 = C(T_1|_X)- C(T_2|_X)$ and $V_2 = C(T_2|_X) - C(T_1|_X)$, and $E = \{(\pi, \pi') \mid \pi \in V_1, \pi' \in V_2$, $\pi$ is not compatible with $\pi'\}$
        \State compute the maximum weight independent set $I$ in $G$ with weight $w$
        \State let $H(\hat{v}) = \ntriv \cap (C(T_1|_X) \cup C(T_2|_X))$
        \State let $R(\hat{v}) = \emptyset$
        \For{each $\pi \in \ntriv \cap (C(T_1|_X) \cup C(T_2|_X))$} 
            \State $sv(\pi) = \hat{v}$
            \State add the root of each $t\in \mathcal{T}(\pi)$ to $R(v)$
        \EndFor
        \For{each $\pi \in \ntriv \cap (I \cup (C(T_1|_X) \cap C(T_2|_X)))$}
            \State $T \gets $ Refine($T_1,T_2, T, \pi, H, sv, \mathcal{T}$)
        \EndFor
        \State refine $T$ arbitrarily at polytomies until it is fully resolved
        \State return $T$
    \end{algorithmic}
\end{algorithm}

For the analysis of the algorithm, we differentiate between two kinds of bipartitions in $C(T_1) \cup C(T_2)$. Let $\Pi_Y = \{\pi = A|B \in C(T_1) \cup C(T_2) \mid \text{either } A\cap X = \emptyset \text{, or } B \cap X = \emptyset\}$. Let $\Pi_X = \{\pi = A|B \in C(T_1) \cup C(T_2) \mid A\cap X \neq \emptyset \text{ and } B\cap X \neq \emptyset \}$. Intuitively, $\Pi_X$ is the set of bipartitions in $C(T_1)\cup C(T_2)$ that are induced by edges in the backbone trees $T_1|_X$ and $T_2|_X$ while $\Pi_Y$ is the set of bipartitions in $C(T_1)\cup C(T_2)$ that are induced by edges inside or connecting extra subtrees of $T_1$ and $T_2$. It follows by definition that $\Pi_X$ and $\Pi_Y$ is a disjoint decomposition of $C(T_1) \cup C(T_2)$. \\

Let $p_X(T)$ and $p_Y(T)$ (we omit the parameters $T_1$ and $T_2$ for brevity) be the contributions to the support score of $T$ from bipartitions of $\Pi_X$ and $\Pi_Y$ for any $T \in \mathcal{T}_S$, respectively. Formally, we have 
\begin{align*}
    p_X(T) &= |C(T|_{S_1}) \cap C(T_1) \cap \Pi_X| + |C(T|_{S_2}) \cap C(T_2) \cap \Pi_X|,\\
    p_Y(T) &= |C(T|_{S_1}) \cap C(T_1) \cap \Pi_Y| + |C(T|_{S_2}) \cap C(T_2) \cap \Pi_Y|.
\end{align*}
By definition of support score, any bipartition can only contribute to the support score if it is in $C(T_1) \cup C(T_2)$. Thus, the support score of $T$ equals $p_X(T) + p_Y(T)$ for any tree $T$ on leaf set $S$. Therefore, it is enough for us to show that Algorithm \ref{alg:maxbisup} finds a tree $T$ that maximizes both $p_X(T)$ and $p_Y(T)$ at the same time.

\begin{lemma}
    For any tree $T$ of leaf set $S$ and any refinement $T'$ of $T$, $p_X(T')\ge p_X(T)$ and $p_Y(T') \ge p_Y(T)$.
\end{lemma}

\begin{lemma}\label{lem:max_pY}
    For any tree $T$ of leaf set $S$, $p_Y(T) \le |\Pi_Y|$. In particular, let $\hat{T}$ be the tree constructed in Algorithm \ref{alg:maxbisup}. Then, $p_Y(\hat{T}) = |\Pi_Y|$. 
\end{lemma}

\begin{claim} \label{claim:begin}
    Let $\hat{T}$ be the tree constructed in Algorithm \ref{alg:maxbisup}, then $p_X(\hat{T}) = 2 |X|$. 
\end{claim}

\begin{lemma} \label{lem:edge_contri}
    Let $\pi = A|B$ be a bipartition of $X$. Let $T$ be a tree of leaf set $S$ such that $\pi \notin C(T|_X)$ and all bipartitions in $C(T|_X)$ are compatible with $\pi$. Let $T'$ be a refinement of $T$ such that for all $\pi' \in C(T'|_{S_i}) \backslash C(T|_{S_i})$ for some $i \in \{1,2\}$, $\pi'|_X = \pi$. Then, $p_X(T') - p_X(T) \le w(\pi)$. 
\end{lemma}

\begin{lemma}
    For any compatible set $F$ of bipartitions of $X$, let $T$ be a tree of leaf set $S$ such that $C(T|_X) = F$. Then $p_X(T) \le \sum_{\pi \in F} w(\pi)$.
\end{lemma}

\begin{lemma}
	$p_X(T^*) = \sum_{}$
\end{lemma}


\bibliographystyle{plain}
\bibliography{references}
\end{document}