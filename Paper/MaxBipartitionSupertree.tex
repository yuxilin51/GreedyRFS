\documentclass{article}
\usepackage[utf8]{inputenc}
\usepackage{amsmath, amsfonts, amssymb,amsthm, mathtools,xcolor, algorithm, algorithmicx, dsfont, cite}
\usepackage[noend]{algpseudocode}

\newcommand{\note}[1]{\textcolor{red}{#1}}
\newcommand{\opt}{\mathrm{OPT}}
\newcommand{\In}{\mathrm{In}}
\newcommand{\triv}{\mathrm{Triv}}
\newcommand{\ntriv}{\mathrm{NonTriv}}
\DeclareMathOperator*{\argmin}{argmin}
\DeclareMathOperator*{\argmax}{argmax}
\DeclareMathOperator*{\RF}{RF}
\newcommand{\bs}{bisup}
\DeclareMathOperator*{\extra}{Extra}
\setlength\parindent{0pt}
\newtheorem{theorem}{Theorem}
\newtheorem{lemma}{Lemma}
\newtheorem{claim}{Claim}
\newtheorem{corollary}{Corollary}
\newtheorem{definition}{Definition}

\title{Minimum Robinson-Foulds Distance Supertree}
\author{Xilin Yu, Thien Le, Sarah Christensen,  Erin Molloy, Tandy Warnow}
\date{\today}

\begin{document}

\maketitle

\section{Introduction}






\section{The Maximum Bipartition Support Supertree Problem} \label{sec:alg}
\subsection{Terminology and Preliminary}

Throughout the paper, we consider only unrooted trees. For any tree $T$, let $V(T)$, $E(T)$, and $L(T)$ denote the vertex set, the edge set, and the leaf set of $T$, respectively. For any $v\in V(T)$, let $N_T(v)$  A tree is \textit{fully resolved} if every non-leaf node has degree $3$. Let $\mathcal{T}_S$ denote the set of all fully resolved trees on leaf set $S$. In any tree $T$, each edge $e$ induces a bipartition $\pi_e := A|B$ of the leaf set, where $A$ and $B$ are the leaves in the two components of $T-e$, respectively. A bipartition $A|B$ is non-trivial if both sides have size at least $2$. For a tree $T$, $C(T) := \{\pi_e \mid e\in E(T)\}$ denotes the set of all bipartitions of $T$. For a fully resolved tree with $n$ leaves, $C(T)$ contains $2n-3$ bipartitions, exactly $n-3$ of which are non-trivial. A tree $T'$ is a \textit{refinement} of $T$ if $T$ can be obtained from $T'$ by contracting a set of edges. Equivalently, $T'$ is a refinement of $T$ if and only if $C(T) \subseteq C(T')$.\\

Two bipartitions $\pi_1$ and $\pi_2$ of the same leaf set are \textit{compatible} if and only if there exists a tree $T$ such that $\pi_1, \pi_2 \in C(T)$. The following theorem and corollary give other categorizations of compatibility.
\begin{theorem}[Theorem 2.20 of \cite{warnow2017computational}]\label{thm:compatibility}
    A pair of bipartitions $A|B$ and $A'|B'$ of the same set is compatible if and only if at least one of the four pairwise intersections $A \cap A'$, $A\cap B'$, $B\cap A'$, $B \cap B'$ is empty. 
\end{theorem}

\begin{corollary}\label{cor:compatibility}
     A pair of bipartitions $A|B$ and $A'|B'$ of the same set is compatible if and only if one side of $A|B$ is a subset of one side of $A'|B'$.
\end{corollary}
\medskip

A tree $T$ restricted to a subset $R$ of its leaf set, denoted $T|_R$, is the minimal subtree of $T$ spanning $R$ with nodes of degree two suppressed. A bipartition $\pi = A|B$ restricted to a subset $R \subseteq A\cup B$ is $\pi|_R = A\cap R | B\cap R$. We have the following intuitive lemma with its proof in the appendix.

\begin{lemma} \label{lem:bipar_restrict_edge}
    Let $T$ be a tree with leaf set $S$ and let $\pi = A|B \in C(T)$ be a bipartition induced by $e \in E(T)$. Let $R \subseteq S$.
    \begin{enumerate}
        \item If $R \cap A = \emptyset$ or $R \cap B = \emptyset$, then $e \notin E(T|_R)$.
        \item If $R \cap A \neq \emptyset$ and $R \cap B \neq \emptyset$, then for any $\pi' \in C(T|_R)$ induced by $e' \in E(T|_R)$, $\pi|_R = \pi'$ if and only if $e \in P(e')$.
    \end{enumerate}
\end{lemma}

\begin{corollary} \label{cor:bipar_restrict}
    Let $T$ be a tree with leaf set $S$ and let $\pi = A|B \in C(T)$ be a bipartition induced by $e \in E(T)$. Let $R \subseteq S$ such that $R \cap A \neq \emptyset$ and $R \cap B \neq \emptyset$. Then $\pi|_R \in C(T|_R)$. 
\end{corollary}


\begin{definition}
For two trees $T$, $T'$ with the same leaf set, the \textit{bipartition support} of them is $\bs(T, T') := |C(T) \cap C(T')|$.
\end{definition}

Bipartition support measures the similarity between the topology of the trees.\\

\subsection{Problem Statement}
Let $T_1$ and $T_2$ be two fully resolved trees on leaf sets $S_1$ and $S_2$, respectively, such that $X := S_1 \cap S_2 \neq \emptyset$. Let $S := S_1 \cup S_2$. The Maximum Bipartition Support Supertree problem, abbreviated \textsc{Max-Bisup-Supertree}, finds a fully resolved supertree $T^*$ on leaf set $S$ that maximizes the sum of the bipartition support of $T^*$ with respect to $T_1$ and $T_2$. That is, 
\begin{align*}
    T^* &= \argmax_{T \in \mathcal{T}_S} \bs(T|_{S_1}, T_1) + \bs(T|_{S_2}, T_2)\\ 
        &= \argmax_{T \in \mathcal{T}_S} |C(T|_{S_1})\cap C(T_1)| + |C(T|_{S_2}) \cap C(T_2)|.
\end{align*}
We call $\bs(T|_{S_1}, T_1) + \bs(T|_{S_2}, T_2)$ the support score of $T$ when $T_1$ and $T_2$ are clear from context.

\subsection{Algorithm}

We first set up the notations for the algorithm and the analysis. Let $T_1,T_2,S_1,S_2$, and $X$ be defined as from the problem statement. Let $T_1|_X$ and $T_2|_X$ be the backbone trees of $T_1$ and $T_2$, respectively. Let $\Pi$ be the set of bipartitions of $X$. Let $\triv$ and $\ntriv$ denotes the set of trivial and non-trivial bipartitions in $C(T_1|_X) \cup C(T_2|_X)$. For each $e \in E(T_i|_X)$, $i \in \{1,2\}$, let $P(e)$ denote the path in $T_i$ from which $e$ is obtained by suppressing all degree-two nodes. Let $w(e)$ be the number of edges on $P(e)$. \\

We define a weight function $w:\Pi \to \mathbb{N}_{\ge 0}$ such that for any bipartition $\pi$ of $X$, $w(\pi) = w(e_1) + w(e_2)$, where $e_i$ induces $\pi$ in $T_i|_X$ for $i \in \{1,2\}$. If for any $i \in \{1,2\}$, no $e_i$ exists that induces $\pi$ in $T_i|_X$, then we use $w(e_i) = 0$.\\

For each $i \in \{1,2\}$ and each $e \in E(T_i|_X)$, let $\In(e)$ be the set of internal nodes of $P(e)$. For each $v \in \In(e)$, let $L(v)$ be the set of leaves in $S_i \backslash X$ whose connecting path to the backbone tree $T_i|_X$ goes through $v$ and let $T(v)$ be the minimal subtree spanning $L(v)$ in $T_i$. We say $T(v)$ is an extra subtree attached to $v$. Consider $T(v)$ rooted at the node $u$ which is the neighbor of $v$ in $T(v)$. Let $\mathcal{T}(e) := \{T(v) \mid v \in \In(e)\}$. Then $\mathcal{T}(e)$ is the set of extra subtrees attached to internal nodes of $P(e)$ in $T_i$. We note that $|\mathcal{T}(e)| = |\In(e)| = w(e)-1$. For any bipartition $\pi \in C(T_1|_X) \cup C(T_2|_X)$, we denote $\mathcal{T}(\pi) := \mathcal{T}(e_1) \cup \mathcal{T}(e_2)$, where $e_i$ is the edge that induces $\pi$ in $T_i|_X$ for $i \in \{1,2\}$ if $\pi \in C(T_i|_X)$. Let $\extra(T_i) := \bigcup_{e \in E(T_i|_X)} \mathcal{T}(e)$. Then $\extra := \extra(T_1) \cup \extra(T_2)$ denotes the set of all extra subtrees in $T_1$ and $T_2$. \note{figure to help}

\begin{algorithm}
    \caption{Max-BiSup Supertree}
    \label{alg:maxbisup}
    \textbf{Input}: two fully resolved trees $T_1$, $T_2$ with leaf sets $S_1$ and $S_2$ where $S_1 \cap S_2 = X \neq \emptyset$\\
    \textbf{Output}: a fully resolved supertree $T$ on leaf set $S = S_1 \cup S_2$ that maximizes the support score 
    \begin{algorithmic}[1]
        \State compute $C(T_1|_X)$ and $C(T_2|_X)$
        \For{each $\pi \in C(T_1|_X) \cup C(T_2|_X)$}
            \State compute $\mathcal{T}(\pi)$ and $w(\pi)$
        \EndFor
        \State construct $T$ by having a star of leaf set $X$ with center vertex $\hat{v}$ and connecting the root of each $t \in \extra$ to $\hat{v}$, let $\hat{T} = T$       
        \For{each $\pi \in \triv$}
            \State $T \gets $ Refine-Triv($T_1, T_2, T, \pi, \hat{v}, \mathcal{T}$)
        \EndFor
        \State let $\tilde{T} = T$
        \State construct the incompatibility graph $G = (V_1 \cup V_2, E)$, where $V_1 = C(T_1|_X)- C(T_2|_X)$ and $V_2 = C(T_2|_X) - C(T_1|_X)$, and $E = \{(\pi, \pi') \mid \pi \in V_1, \pi' \in V_2$, $\pi$ is not compatible with $\pi'\}$
        \State compute the maximum weight independent set $I$ in $G$ with weight $w$
        \State let $H(\hat{v}) = \ntriv \cap (C(T_1|_X) \cup C(T_2|_X))$
        \For{each $\pi \in \ntriv \cap (C(T_1|_X) \cup C(T_2|_X))$} 
            \State $sv(\pi) = \hat{v}$
        \EndFor
        \For{each $\pi \in \ntriv \cap (I \cup (C(T_1|_X) \cap C(T_2|_X)))$}
            \State $T \gets $ Refine($T_1,T_2, T, \pi, H, sv, \mathcal{T}$)
        \EndFor
        \State let $T^* = T$
        \State refine $T$ arbitrarily at polytomies until it is fully resolved
        \State return $T$
    \end{algorithmic}
\end{algorithm}

\begin{algorithm}
    \caption{Refine-Triv}
    \label{alg:trivial_refine}

    \textbf{Input}: two trees $T_1$, $T_2$ with leaf sets $S_1$ and $S_2$ where $S_1 \cap S_2 = X \neq \emptyset$, an unrooted tree $T$ on leaf set $S = S_1 \cup S_2$, a trivial bipartition $\pi = A|b$ of $X$, a vertex $\hat{v} \in V(T)$, a dictionary $\mathcal{T}$\\
    \textbf{Output}: an tree $T'$ which is a refinement of $T$ such that $\pi \in C(T'|_X)$ 
    \begin{algorithmic}[1]
        \State detach all extra subtrees in $\mathcal{T}(\pi)$ from $\hat{v}$ and attach them onto $(\hat{v},b)$ such that the subtrees from $\mathcal{T}(e_1)$ and subtrees from $\mathcal{T}(e_2)$ are side by side and each group respects the ordering of subtrees in $T_i$
        \State return the resulting tree $T'$
    \end{algorithmic}
\end{algorithm}

\begin{algorithm}
    \caption{Refine}
    \label{alg:refine}
    \textbf{Input}: two trees $T_1$, $T_2$ with leaf sets $S_1$ and $S_2$ where $S_1 \cap S_2 = X \neq \emptyset$, an unrooted tree $T$ on leaf set $S = S_1 \cup S_2$, a bipartition $\pi = A|B$ of $X$, a dictionary $H$, a dictionary $sv$, a dictionary $\mathcal{T}$\\
    \textbf{Output}: an tree $T'$ which is a refinement of $T$ such that $\pi \in C(T'|_X)$ 
    \begin{algorithmic}[1]
        \State $v \gets sv(\pi)$
        \State compute $N_A:= \{u \in N_T(v) \mid \text{$\exists a \in A$ such that $u$ can reach $a$ in $T-v$}\}$ and $N_B:= \{u \in N_T(v) \mid \text{$\exists b \in B$ such that $u$ can reach $b$ in $T-v$}\}$.
        \State $V(T) \gets V(T) \cup \{v_a, v_b\}$, $E(T) \gets E(T) \cup \{(v_a,v_b)\}$
        \State $H(v_a) \gets \emptyset, H(v_b) \gets \emptyset$
        \For{each $u \in N_A \cup N_B$} 
            \If{$u \in N_A$} connect $u$ to $v_a$
            \Else{} connect $u$ to $v_b$
            \EndIf
        \EndFor
        \State detach all extra subtrees in $\mathcal{T}(\pi)$ from $v$ and attach them onto $(v_a,v_b)$ such that the subtrees from $\mathcal{T}(e_1)$ and subtrees from $\mathcal{T}(e_2)$ are side by side and each group respects the ordering of subtrees in $T_i$
        \For{each bipartition $\pi'= A'|B' \in H(v)$ such that $\pi' \neq \pi$}
            \State detach all extra subtrees in $\mathcal{T}(\pi')$ from $v$ 
            \If{$A' \subseteq A$ or $B' \subseteq A$}
                \State $sv(\pi') = v_a$ and $H(v_a) \gets H(v_a) + \pi'$
                \State attach all extra subtrees in $\mathcal{T}(\pi')$ to $v_a$
            \ElsIf{$A' \subseteq B$ or $B' \subseteq B$}
                \State $sv(\pi') = v_b$ and $H(v_b) \gets H(v_b) + \pi'$
                \State attach all extra subtrees in $\mathcal{T}(\pi')$ to $v_b$
            \Else{} 
                \State discard $\pi'$ and attach all extra subtrees in $\mathcal{T}(\pi')$ to either $v_a$ or $v_b$ 
            \EndIf
        \EndFor
        \For{each remaining extra subtree attached to $v$}
            \State detach it from $v$ and attach it to either $v_a$ or $v_b$
        \EndFor
        \State delete $v$ and incident edges from $T$
        \State return the resulting tree $T'$
    \end{algorithmic}
\end{algorithm}

For the analysis of the algorithm, we differentiate between two kinds of bipartitions in $C(T_1) \cup C(T_2)$. Let $\Pi_Y = \{\pi = A|B \in C(T_1) \cup C(T_2) \mid \text{either } A\cap X = \emptyset \text{, or } B \cap X = \emptyset\}$. Let $\Pi_X = \{\pi = A|B \in C(T_1) \cup C(T_2) \mid A\cap X \neq \emptyset \text{ and } B\cap X \neq \emptyset \}$. Intuitively, $\Pi_X$ is the set of bipartitions in $C(T_1)\cup C(T_2)$ that are induced by edges in the backbone trees $T_1|_X$ and $T_2|_X$ while $\Pi_Y$ is the set of bipartitions in $C(T_1)\cup C(T_2)$ that are induced by edges inside extra subtrees or connecting extra subtrees to the backbone trees.\\

Let $p_X(T)$ and $p_Y(T)$ (we omit the parameters $T_1$ and $T_2$ for brevity) be the contributions to the support score of $T$ from bipartitions of $\Pi_X$ and $\Pi_Y$ for any $T \in \mathcal{T}_S$, respectively. Formally, we have 
\begin{align*}
    p_X(T) &= |C(T|_{S_1}) \cap C(T_1) \cap \Pi_X| + |C(T|_{S_2}) \cap C(T_2) \cap \Pi_X|,\\
    p_Y(T) &= |C(T|_{S_1}) \cap C(T_1) \cap \Pi_Y| + |C(T|_{S_2}) \cap C(T_2) \cap \Pi_Y|.
\end{align*}

\begin{claim}
    If Algorithm \ref{alg:maxbisup} returns a tree $T$ such that $p_X(T) \ge p_X(T')$ and $p_Y(T) \ge p_Y(T')$ for any tree $T'$ with leaf set $S$, then Algorithm \ref{alg:maxbisup} solves \textsc{Max-Bisup-Supertree} correctly.
\end{claim}
\begin{proof}
By definition of support score, any bipartition can only contribute to the support score if it is in $C(T_1) \cup C(T_2)$. It follows by definition of $\Pi_X$ and $\Pi_Y$ that $\Pi_X$ and $\Pi_Y$ is a disjoint decomposition of $C(T_1) \cup C(T_2)$. Thus, the support score of $T$ equals $p_X(T) + p_Y(T)$ for any tree $T$ on leaf set $S$. Then if $p_X(T) \ge p_X(T')$ and $p_Y(T) \ge p_Y(T')$ for any tree $T'$ with leaf set $S$, $T$ achieves the maximum support score among all trees of leaf set $S$, in particular, it achieves the maximum support score among all trees in $\mathcal{T}_S$.
\end{proof}
Therefore, it is enough for us to show that Algorithm \ref{alg:maxbisup} finds a tree $T$ that maximizes both $p_X(T)$ and $p_Y(T)$ at the same time.


\begin{lemma}\label{lem:refine_only_increases}
    For any tree $T$ of leaf set $S$ and any refinement $T'$ of $T$, $p_X(T')\ge p_X(T)$ and $p_Y(T') \ge p_Y(T)$.
\end{lemma}
\begin{proof}
    Since $T'$ is an refinement of $T$, $C(T|_{S_i}) \subseteq C(T'|_{S_i})$ for any $i \in \{1,2\}$. Therefore, $|C(T|_{S_i}) \cap C(T_i) \cap \Pi_X| \le |C(T'|_{S_i}) \cap C(T_i) \cap \Pi_X|$ for any $i \in \{1,2\}$, and thus $p_X(T) \le p_X(T')$. Similarly, $|C(T|_{S_i}) \cap C(T_i) \cap \Pi_Y| \le |C(T'|_{S_i}) \cap C(T_i) \cap \Pi_Y|$ for any $i \in \{1,2\}$, and thus $p_Y(T) \le p_Y(T')$.
\end{proof}


\begin{lemma}\label{lem:max_pY}
    For any tree $T$ of leaf set $S$, $p_Y(T) \le |\Pi_Y|$. In particular, let $\hat{T}$ be the tree constructed in Algorithm \ref{alg:maxbisup}. Then, $p_Y(\hat{T}) = |\Pi_Y|$. 
\end{lemma}
\begin{proof}
    Since $T_1$ and $T_2$ has different leaf sets, $C(T_1)$ and $C(T_2)$ are disjoint. Since $\Pi_Y \subseteq C(T_1)\cup C(T_2)$, $C(T_1) \cap \Pi_Y$ and $C(T_2)\cap \Pi_Y$ forms a disjoint decomposition of $\Pi_Y$. By definition of $p_Y(\cdot)$, for any tree $T$ of leaf set $S$,
    \begin{align*}
        p_Y(T) &= |C(T|_{S_1}) \cap C(T_1) \cap \Pi_Y| + |C(T|_{S_2}) \cap C(T_2) \cap \Pi_Y| \\
        &\le |C(T_1) \cap \Pi_Y| + | C(T_2) \cap \Pi_Y| \\
        &= |\Pi_Y|.
    \end{align*}
    Fix any $\pi = A|B \in \Pi_Y$. By definition of $\Pi_Y$, either $A \cap X = \emptyset$ or $B \cap X = \emptyset$. Assume without loss of generality that $A \cap X = \emptyset$. If $\pi \in C(T_1)$, let $e_1$ be the edge that induces $\pi$ in $T_1$. Then $A \subseteq S_1 \backslash X$, which implies either $e_1$ is an internal edge in an extra subtree in $\extra(T_1)$, or $e_1$ connects one extra subtree in $\extra(T_1)$ to the backbone $T_1|_X$. In either case, the construction of $\hat{T}$ ensures that $\pi \in C(\hat{T}|_{S_1})$. Similarly if $\pi \in C(T_2)$, then $\pi \in C(\hat{T}|_{S_2})$ by construction. Therefore, each bipartition $\pi \in \Pi_Y$ contributes $1$ to $|C(\hat{T}|_{S_i}) \cap C(T_i) \cap \Pi_Y|$ for exactly one $i \in \{1,2\}$ and thus it contributes $1$ to $p_Y(\hat{T})$. Hence, $p_Y(\hat{T}) = |\Pi_Y|$.\\
\end{proof}

\begin{claim} \label{claim:begin}
    Let $\hat{T}$ be the tree constructed in Algorithm \ref{alg:maxbisup}, then $p_X(\hat{T}) = 2 |X|$. 
\end{claim}
\begin{proof}
    Let the center of the star from which $\hat{T}$ is constructed be the center of $\hat{T}$. For each $v \in X$, consider the bipartition $\pi_v = \{v\}\mid S \backslash \{v\}$ induced by the edge that connects the leaf $v$ to the center. It is easy to see that $\pi_v|_{S_i} = \{v\} \mid S_i \backslash \{v\} \in C(T_i)\cap C(\hat{T}|_{S_i})$ for any $i\in\{1,2\}$ as $\pi_v|_{S_i}$ is a trivial bipartition of $S_i$ and must be present in any tree on leaf set $S_i$. We also know $\pi_v|_{S_i} \in \Pi_X$ as $\pi_v \in C(T_1)\cup C(T_2)$ and both sides of $\pi_v$ has non-empty intersection with $X$. Thus, $\pi_v|_{S_i} \in C(\hat{T}|_{S_i}) \cap C(T_i) \cap \Pi_X$ for any $i \in \{1,2\}$. So for each $v \in X$, $\pi_v|_{S_1}$ and $\pi_v|_{S_2}$ each contributes $1$ to $p_X(\hat{T})$. Therefore, $p_X(\hat{T}) \ge 2|X|$. \\
    
    Fix any bipartition $\pi = A|B$ induced by any other edge of $\hat{T}$ such that $\pi|_{S_i} \in C(\hat{T}|_{S_i})$ for some $i \in \{1,2\}$. By construction of $\hat{T}$, the edge inducing $\pi$ is either inside an extra subtree or connecting the root of an extra subtree to the center Therefore, either $A \subseteq S\backslash X $ or $B \subseteq S \backslash X$, which implies $\pi|_{S_i} \notin \Pi_X$ for any $i \in \{1,2\}$. Hence, there is no other bipartition of $\hat{T}$ such that when restrict to $S_i$ contributes to $p_X(\hat{T})$. Therefore, $p_X(\hat{T}) = 2|X|$.
\end{proof}




\begin{lemma} \label{lem:one_bipar_upperbound}
    Let $\pi = A|B$ be a bipartition of $X$. Let $T$ be a tree of leaf set $S$ such that $\pi \notin C(T|_X)$ and all bipartitions in $C(T|_X)$ are compatible with $\pi$. Let $T'$ be a refinement of $T$ such that for all $\pi' \in C(T'|_{S_i}) \backslash C(T|_{S_i})$ for some $i \in \{1,2\}$, $\pi'|_X = \pi$. Then, $p_X(T') - p_X(T) \le w(\pi)$. 
\end{lemma}
\begin{proof}
    By definition of $p_X(\cdot)$, 
    \begin{align*}
        & p_X(T') - p_X(T) \\
        =& |C(T'|_{S_1}) \cap C(T_1) \cap \Pi_X| + |C(T'|_{S_2}) \cap C(T_2) \cap \Pi_X| \\
        &- (|C(T|_{S_1}) \cap C(T_1) \cap \Pi_X| + |C(T|_{S_2}) \cap C(T_2) \cap \Pi_X|) \\
        =& |(C(T'|_{S_1})\backslash C(T|_{S_1})) \cap C(T_1) \cap \Pi_X|+|(C(T'|_{S_2})\backslash C(T|_{S_2})) \cap C(T_2) \cap \Pi_X|\\
        =& \sum_{i = 1,2}|(C(T'|_{S_i})\backslash C(T|_{S_i})) \cap C(T_i) \cap \Pi_X|.
    \end{align*}
    Therefore, we only need to prove that $\sum_{i = 1,2}|(C(T'|_{S_i})\backslash C(T|_{S_i})) \cap C(T_i) \cap \Pi_X| \le w(\pi)$. For any $\pi' \in (C(T'|_{S_i})\backslash C(T|_{S_i})) \cap C(T_i) \cap \Pi_X$ for any $i \in \{1,2\}$, we have $\pi'|_X = \pi$. \\
    
    We differentiate three different cases for the proof of the above statement: 1) $\pi \notin C(T_1|_X)\cup C(T_2|_X)$, 2) $\pi \in C(T_1|_X) \Delta C(T_2|_X)$, 3) $\pi \in C(T_1|_X) \cap C(T_2|_X)$. \\
    
    Case 1): Let $\pi \notin C(T_1|_X)\cup C(T_2|_X)$. Since no edge induces $\pi$ in $T_1|_X$ or $T_2|_X$, we have $w(\pi) = 0$. Assume for contradiction that there exists a bipartition $\pi'\in (C(T'|_{S_i})\backslash C(T|_{S_i})) \cap C(T_i) \cap \Pi_X$ for some $i \in \{1,2\}$.  Since $\pi \notin C(T_1|_X)\cup C(T_2|_X)$ and $\pi'|_X = \pi$, by Corollary \ref{cor:bipar_restrict}, $\pi' \notin C(T_i)$ for any $i \in \{1,2\}$. This contradicts with the fact that $\pi' \in C(T_i)$ for some $i \in \{1,2\}$. Therefore, the assumption that there exists such a bipartition $\pi'$ is wrong and $\sum_{i = 1,2}|(C(T'|_{S_i})\backslash C(T|_{S_i})) \cap C(T_i) \cap \Pi_X| = 0 \le w(\pi)$.\\
    
    Case 2): Let $\pi \in C(T_1|_X) \Delta C(T_2|_X)$. Assume without loss of generality that $\pi \in C(T_1|_X) \backslash C(T_2|_X)$. Then, we have $w(\pi) = w(e_1)$. Let $\pi'\in (C(T'|_{S_i})\backslash C(T|_{S_i})) \cap C(T_i) \cap \Pi_X$ for some $i \in \{1,2\}$. Since $\pi'|_X = \pi$ and $\pi \notin C(T_2|_X)$, by Corollary \ref{cor:bipar_restrict}, we have $\pi' \notin C(T_2)$. Since $\pi' \in C(T_i)$ for some $i\in \{1,2\}$, it must be that $\pi' \in C(T_1)$. By Lemma \ref{lem:bipar_restrict_edge}, the edge which induces $\pi'$ in $T_1$ is an edge on $P_1(e_1)$. Since there are $w(e_1)$ edges on $P_1(e_1)$, there are at most $w(e_1)$ distinct such bipartitions $\pi'$s, and thus the statement is proved.\\
    
    Case 3): Let $\pi \in C(T_1|_X) \cap C(T_2|_X)$. Then we have $w(\pi) = w(e_1)+w(e_2)$. Fix any $\pi'\in (C(T'|_{S_1})\backslash C(T|_{S_1})) \cap C(T_1) \cap \Pi_X$. Since $\pi' \in C(T_1)$ and $\pi'|_X = \pi \in C(T_1|_X)$, by Lemma \ref{lem:bipar_restrict_edge}, the edge $e'$ that induces $\pi'$ is an edge on $P_1(e_1)$. Recall that $w(e_1) = |P_1(e_1)|$, then we have $|(C(T'|_{S_1})\backslash C(T|_{S_1})) \cap C(T_1) \cap \Pi_X| \le |P_1(e_1)| = w(e_1)$. Similarly, $|(C(T'|_{S_2})\backslash C(T|_{S_2})) \cap C(T_2) \cap \Pi_X| \le |P_2(e_2)| = w(e_2)$. Therefore, $\sum_{i = 1,2}|(C(T'|_{S_i})\backslash C(T|_{S_i})) \cap C(T_i) \cap \Pi_X| \le w(\pi)$.
\end{proof}


\begin{lemma} \label{lem:compatible_set_upperbound}
    For any compatible set $F$ of bipartitions of $X$, let $T$ be a tree of leaf set $S$ such that $C(T|_X) = F$. Then $p_X(T) \le \sum_{\pi \in F} w(\pi)$.
\end{lemma}
\begin{proof}
    
    Fix an arbitrary ordering of bipartitions in $F$ and let them be $\pi_1,\pi_2,\dots,\pi_k$, where $k = |F|$. Let $F_i = \{\pi_1,\dots, \pi_i\}$ for any $i \in \{0,1,\dots,k\}$. In particular, $F_0 = \emptyset$ and $F_k = F$. Let $T^i$ be obtained by contracting any edge $e$ in $T$ such that $\pi_e \in \Pi_X$ and $\pi_e|_X \notin F_i$. Then $C(T^i|_X) = F_i$. In particular, we know $C(T^0|_X) = \emptyset$. By construction, $T^i$ is a refinement of $T^{i-1}$ for any $i \in \{1,2,\dots,k\}$ such that for any $\pi' \in C(T^i)\backslash C(T^{i-1})$, $\pi'|_X = \pi_i$. Then by Lemma \ref{lem:one_bipar_upperbound}, $p_X(T^i) - p_X(T^{i-1}) \le w(\pi_i)$. Therefore, 
    \[p_X(T) - p_X(T^0) = \sum_{i = 1}^k p_X(T^i) - p_X(T^{i-1}) \le \sum_{i \in F}w(\pi_i).\]
    
    We also know that $p_X(T^0) = 0$ \note{(expand on this)} and thus $p_X(T) \le \sum_{i \in I}w(\pi_i)$ as desired.
\end{proof}

\begin{claim}\label{claim:after_add_trivial}
    Let $\tilde{T}$ be the tree constructed in Algorithm \ref{alg:maxbisup}, then $p_X(\tilde{T}) = \sum_{\pi \in \triv} w(\pi)$. 
\end{claim}


\begin{lemma}\label{lem:refine_achieves_weight}
\end{lemma}

Let $G$ be the incompatibility graph defined in Algorithm \ref{alg:maxbisup} and $I$ be the maximum weight independent set in $G$ with weight function $w$. Let $G'= (V_1' \cup V_2',E')$ be another incompatibility graph such that $V_1' = C(T_1|_X)$ and $V_2' = C(T_2|_X)$, and $E' = \{(\pi, \pi') \mid \pi \in V_1', \pi' \in V_2'$, $\pi$ is not compatible with $\pi'\}$.  Let $I' := I \cup (C(T_1|_X )\cap C(T_2|_X))$.


\begin{claim}\label{claim:independent_sets_equivalence}
    $I'$ is a maximum weight independent set in $G'$ with weight function $w$. 
\end{claim}
\begin{proof}
\end{proof}

\begin{claim}\label{claim:after_add_all_bipars}
    Let $T^*$ be the tree defined in Algorithm \ref{alg:maxbisup}, $p_X(T^*) \ge p_X(T)$ for any tree $T$ of leafset $S$.
\end{claim}
\begin{proof}
then $p_X(T^*)= \sum_{\pi \in I} w(\pi) + \sum_{\pi \in (C(T_1|_X )\cap C(T_2|_X))} w(\pi) = \sum_{\pi \in I'}w(\pi)$.
\end{proof}


\begin{corollary}\label{cor:maximization}
    Let $T^*$ be the tree defined in Algorithm \ref{alg:maxbisup}, $p_X(T^*) \ge p_X(T)$ for any tree $T$ of leafset $S$.
\end{corollary}

\begin{theorem}\label{thm:correctness_alg}
Algorithm \ref{alg:maxbisup} correctly solves \textsc{Max-Bisup-Supertree} in $O(n^3)$ \note{(Checking this)} time.
\end{theorem}
\begin{proof}
\end{proof}

\begin{theorem}\label{thm:hardness}
\end{theorem}

\appendix
\section{Proofs from Section \ref{sec:alg}}

Proof of Lemma \ref{lem:bipar_restrict_edge}
\begin{proof}
Let $T_R$ be the minimal subtree of $T$ that spans $R$. It follows that the leaf set of $T_R$ is $R$ and $T|_R$ is obtained from $T_R$ by suppressing all degree-two nodes. Let $\pi' = A'|B'$.
By definition of $e$ inducing $\pi = A|B$, the vertices of $A$ are all disconnected from vertices of $B$ in $T-e$. If $R\cap A \neq \emptyset$ and $R\cap B \neq \emptyset$, then $e$ is necessary to connect $R\cap A$ with $R \cap B$, and thus $e$ must be in any tree spanning $R$ and in particular $e \in E(T_R)$. Since $T_R$ is a subgraph of $T$, the two components in $T_R-e$ are subgraphs of the two components in $T-e$. Thus, the leaves of the two components in $T_R-e$ are exactly $R\cap A$ and $R\cap B$. We also know that suppressing degree-two nodes does not change the connectivity between any leaves so the leaves of the two components in $T_R - P(e')$ (with vertices on the path also deleted) are the same as the leaves of the two components in $T|_R - e'$, which are $A'$ and $B'$. If $e \in P(e')$, since all internal nodes of $P(e')$ have degree two with both incident edges on $P(e')$, there is no leaf which exists in any of the two components in $T_R - e$ but does not exists in the corresponding component in $T_R-P(e')$. Therefore, $\pi|_R = R\cap A|R\cap B = A'|B' = \pi'$. If $e \notin P(e')$, then since $e \in E(T_R)$, there must exists $e'' \in E(T|_R)$ such that $e'' \neq e'$ and $e \in P(e'')$. By the arguement above, $\pi|_R = \pi''$ where $\pi''$ is the bipartition induced by $e''$ in $T|_R$. Since $e'' \neq e'$, we know $\pi' \neq \pi''$ and thus $\pi|_R \neq \pi'$. This concludes our proof that $\pi|_R = \pi'$ if and only if $e \in P(e')$. 
\end{proof}

\bibliographystyle{plain}
\bibliography{references}
\end{document}
